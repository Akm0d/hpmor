\chapter{Something to Protect: Professor Quirrell}

\lettrine{T}{he} Sun shone
down on the Scottish green, striking sparks of reflected white from every
passing dewdrop or reflective leaf that happened to position itself correctly,
a clear blue sky for a funeral.

Harry had declined to give the eulogy. He’d declined for the second time.
Professor Flitwick had asked him about it weeks ago in May, to give Harry time
to write his lines before it would become necessary to speak; and Harry had
said no then, too.

So it fell to a sixth-year Gryffindor, Oliver Habryka, who had the
fourth-highest total of Quirrell points among all the students, and who had
been General of an army. The seventeen-year-old boy was tall and not especially
handsome in solid black robes; instead of a red tie, he was wearing a purple
tie such as Professor Quirrell had sometimes favoured.

Speaking, under the circumstances, extempore. The previous eulogies,
written well in advance, had been discarded; Oliver Habryka had a parchment in
his left hand, but he wasn’t looking at it at all.

“Professor Quirrell was very sick,” the tall boy said, his wavering voice
falling into a hush of students, occasionally broken by a muffled sob. “I think
if Professor Quirrell had been able to fight in the fullness of his power,
You-Know-Who couldn’t have beat him easily, and maybe not at all. They say that
David Monroe was the only one that You-Know-Who was ever afraid of, in his day.
But,” Oliver’s voice broke, “Professor Quirrell wasn’t in the fullness of his
power. He was very sick. He had trouble walking by himself. And he went to face
the Dark Lord, alone.”

There was a pause, then, while the students cried for a while.

Oliver wiped away his tears with his sleeve, and spoke again. “We don’t know
exactly what happened,” said Oliver. “I imagine the Dark Lord laughed at him.
Maybe made fun of the Professor, for challenging him when he couldn’t stand up.
Well, \emph{he’s not laughing now,} is he.”

There were fierce nods from the students; all of them that Harry could see,
from Gryffindor to Slytherin.

“Maybe the Dark Lord knew some way of curing Professor Quirrell, You-Know-Who
did come back from the dead after all. Maybe he offered Professor Quirrell his
life if Professor Quirrell would serve him. Professor Quirrell smiled, and told
the Dark Lord it was time for them to play a game called Who’s The Most
Dangerous Wizard In The World.”

\emph{If you don’t know, don’t just make stuff up.} But Harry didn’t say
anything. It was what Lord Voldemort might have tried, it was what Professor
Quirrell might have said back.

“And they aren’t telling us everything,” Oliver said, “but we can guess what
happened next. We all know that Hermione Granger, who was one of the
Professor’s best students, was killed by a troll earlier this year, it must
have been the Dark Lord who made it happen, just like he framed her for the
Blood-Cooling Charm. Professor Quirrell knew the Dark Lord was behind it, so he
stole Miss~Granger’s body and preserved it, kept it safe—”

Couldn’t blame him for that one.

“Then Professor Quirrell went out to face the Dark Lord. The Dark Lord killed
Professor Quirrell. And Hermione Granger came back to life. They say she’s
alive and whole now, and maybe something more. When the Dark Lord tried to
seize her, all that was left of him afterward was his burned robes and his
hands around Miss~Granger’s throat. Just as Harry Potter was protected from the
Killing Curse by his mother’s love and sacrifice, Professor Quirrell willingly
going out, to face, the Dark Lord alone, must have called, Hermione Granger’s
spirit, back from, from wherever, she was—” Oliver’s voice was breaking.

“Not just like that,” Harry said from the front row of seats, his own voice
hoarse. He \emph{had} to say something at this point, before it got out of
control. If it wasn’t already out of control. “David Monroe was a powerful
wizard, more powerful than anyone knew except him and me. I don’t think you can
bring someone back from the dead just by sacrificing yourself. No one should
try doing it that way.”

Such a beautiful story. It should have been true. \emph{It should have been
true.}

“I don’t know very much about the person behind the Professor,” Oliver Habryka
said, after he got himself under control again. “I know David Monroe wasn’t a
happy man. He never could cast a Patronus Charm.”

Tears were gathering in Harry’s eyes again. It wasn’t right, it wasn’t fair,
Voldemort had killed so many people, he should have died along with his
followers, he didn’t deserve special treatment. But it hadn’t just been Harry’s
weakness, it had been the horcruxes, Voldemort \emph{couldn’t} have been killed
outright. So Harry could admit it, he was glad, he was \emph{glad} Professor
Quirrell wasn’t all gone…

“But I, know,” said Oliver, tears glistening on his own cheeks, “Professor
Quirrell, is happy, wherever, he is now.”

On Harry’s left hand, a tiny emerald glowed bright beneath the morning sun.

\emph{Not Heaven, not some faraway star, not a different place but a better
person, I’ll show you, some day I’ll show you how to be happy—}

The tall boy glanced down at a parchment he held in his other hand, the first
time he’d consulted it. “Professor Quirrell,” Oliver said, his voice now
fiercer and faster, “was, by far, the best Professor of Battle Magic that
Hogwarts ever had. Salazar Slytherin couldn’t have been half as good a teacher,
no matter what spells he knew. Professor Quirrell told us at the beginning of
this year that what he taught us would always be our firm foundation in the
arts of Defence. And it will be. Forever. We’ll teach it to the new students
next year, no matter who we have for a professor. The older students will teach
the younger ones. That’s the solution to the curse on the Defence position. We
won’t sit around waiting for authority to teach us. And we’ll make sure that
Professor Quirrell’s teachings never die out of Hogwarts.”

Harry looked at where Professor—no, Headmistress McGonagall—was sitting,
and saw the Headmistress nodding silently, a look that was sad and stern and
proud.

“They haven’t let us see Miss~Granger yet,” Oliver said. His voice quavered.
“The Girl-Who-Revived. But I’ll always think of the Defence Professor when I
see her. His sacrifice lives on in her, just as his teachings live on in us.”
Oliver glanced at where Harry sat, then looked down again at the parchment.
“Here’s to Professor Quirrell, then, the best Slytherin that ever was, what
every Slytherin should be! Three cheers for him!”

“\emph{Huzzah! Huzzah! Huzzah!}”

No one stayed silent this time, not a single student that Harry could see.
