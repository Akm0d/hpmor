\chapter{Conscientiousness}

\lettrinepara[ante=“]{F}{\emph{rigideiro!}}”

\hplettrineextrapara
Harry dipped a finger in the glass of water on his desk. It should have been cool. But lukewarm it was, and lukewarm it had stayed. Again.

Harry was feeling very, very cheated.

There were hundreds of fantasy novels scattered around the Verres household. Harry had read quite a few. And it was starting to look like he had a mysterious dark side. So after the glass of water had refused to cooperate the first few times, Harry had glanced around the Charms classroom to make sure no one was watching, and then taken a deep breath, concentrated, and made himself angry. Thought about the Slytherins bullying Neville, and the game where someone knocked down your books every time you tried to pick them up again. Thought about what Draco Malfoy had said about the ten-year-old Lovegood girl and how the Wizengamot really operated…

And the fury had entered his blood, he had held out his wand in a hand that trembled with hate and said in cold tones “\emph{Frigideiro!}” and absolutely nothing had happened.

Harry had been \emph{gypped.} He wanted to write someone and demand a \emph{refund} on his dark side which clearly \emph{ought} to have irresistible magical power but had turned out to be \emph{defective.}

“\emph{Frigideiro!}” said Hermione again from the desk next to him. Her water was solid ice and there were white crystals forming on the rim of her glass. She seemed to be totally intent on her own work and not at all conscious of the other students staring at her with hateful eyes, which was either (a)~dangerously oblivious of her or (b)~a perfectly honed performance rising to the level of fine art.

“Oh, \emph{very} good, Miss~Granger!” squeaked Filius Flitwick, their Charms Professor and Head of Ravenclaw, a tiny little man with no visible signs of being a past dueling champion. “Excellent! Stupendous!”

Harry had expected to be, in the worst case, second behind Hermione. Harry would have preferred for \emph{her} to be rivalling \emph{him,} of course, but he could have accepted it the other way around.

As of Monday, Harry was headed for the bottom of the class, a position for which he was companionably rivalling all the other Muggle-raised students except Hermione. Who was all alone and rivalless at the top, poor thing.

Professor Flitwick was standing over the desk of one of the other Muggleborns and quietly adjusting the way she was holding her wand.

Harry looked over at Hermione. He swallowed hard. It was the obvious role for her in the scheme of things…”Hermione?” Harry said tentatively. “Do you have any idea what I might be doing wrong?”

Hermione’s eyes lit up with a terrible light of helpfulness and something in the back of Harry’s brain screamed in desperate humiliation.

Five minutes later, Harry’s water did seem noticeably cooler than room temperature and Hermione had given him a few verbal pats on the head and told him to pronounce it more carefully next time and gone off to help someone else.

Professor Flitwick had given her a House point for helping him.

Harry was gritting his teeth so hard his jaw ached and that wasn’t helping his pronunciation.

\emph{I don’t care if it’s unfair competition. I know exactly what I am doing with two extra hours every day. I am going to sit in my trunk and study until I am keeping up with Hermione Granger.}

\later

“Transfiguration is some of the most complex and dangerous magic you will learn at Hogwarts,” said Professor McGonagall. There was no trace of any levity upon the face of the stern old witch. “Anyone messing around in my class will leave and not come back. You have been warned.”

Her wand came down and tapped her desk, which smoothly reshaped itself into a pig. A couple of Muggleborn students gave out small yelps. The pig looked around and snorted, seeming confused, and then became a desk again.

The Transfiguration Professor looked around the classroom, and then her eyes settled on one student.

“Mr~Potter,” said Professor McGonagall. “You only received your schoolbooks a few days ago. Have you started reading your Transfiguration textbook?”

“No, sorry professor,” Harry said.

“You needn’t apologise, Mr~Potter, if you were required to read ahead you would have been told to do so.” McGonagall’s fingers rapped the desk in front of her. “Mr~Potter, would you care to guess whether this is a desk which I Transfigured into a pig, or if it began as a pig and I briefly removed the Transfiguration? If you had read the first chapter of your textbook, you would know.”

Harry’s eyebrows furrowed slightly. “I’d guess it’d be easier to start with a pig, since if it started as a desk, it might not know how to stand up.”

Professor McGonagall shook her head. “No fault to you, Mr~Potter, but the correct answer is that in Transfiguration you do \emph{not} care to guess. Wrong answers will be marked with extreme severity, questions left blank will be marked with great leniency. You must learn to know what you do not know. If I ask you any question, no matter how obvious or elementary, and you answer ‘I’m not sure’, I will not hold it against you and anyone who laughs will lose House points. Can you tell me why this rule exists, Mr~Potter?”

\emph{Because a single error in Transfiguration can be incredibly dangerous.} “No.”

“Correct. Transfiguration is more dangerous than Apparition, which is not taught until your sixth year. Unfortunately, Transfiguration must be learned and practised at a young age to maximise your adult ability. So this is a dangerous subject, and you should be quite scared of making any mistakes, because none of my students have ever been permanently injured and I will be \emph{extremely put out} if you are the first class to \emph{spoil my record}.”

Several students gulped.

Professor McGonagall stood up and moved over to the wall behind her desk, which held a polished wooden board. “There are many reasons why Transfiguration is dangerous, but one reason stands above all the rest.” She produced a short quill with a thick end and used it to sketch letters in red, which she then underlined, using the same marker, in blue:

\hackChXV{Transfiguration is not permanent!}

“Transfiguration is not permanent!” said Professor McGonagall. “Transfiguration is not permanent! Transfiguration is not permanent! Mr~Potter, suppose a student Transfigured a block of wood into a cup of water, and you drank it. What do you imagine might happen to you when the Transfiguration wore off?” There was a pause. “Excuse me, I should not have asked that of you, Mr~Potter, I forgot that you are blessed with an unusually pessimistic imagination—”

“I’m fine,” Harry said, swallowing hard. “So the first answer is that I don’t \emph{know},” the Professor nodded approvingly, “but I \emph{imagine} there might be…wood in my stomach, and in my bloodstream, and if any of that water had been absorbed into my body’s tissues—would it be wood pulp or solid wood or…” Harry’s grasp of magic failed him. He couldn’t understand how wood mapped into water in the first place, so he couldn’t understand what would happen after the water molecules were scrambled by ordinary thermal motions and the magic wore off and the mapping reversed.

McGonagall’s face was stiff. “As Mr~Potter has correctly reasoned, he would become extremely sick and require immediate Flooing to St. Mungo’s Hospital if he was to have any chance of survival. Please turn your textbooks to page 5.”

Even without any sound in the moving picture, you could tell that the woman with horribly discoloured skin was screaming.

“The criminal who originally Transfigured gold into wine and gave it to this woman to drink, ‘in payment of the debt’ as he put it, received a sentence of ten years in Azkaban. Please turn to page 6. That is a Dementor. They are the guardians of Azkaban. They suck away at your magic, your life, and any happy thoughts you try to have. The picture on page 7 is of the criminal ten years later, on his release. You will note that he is dead—yes, Mr~Potter?”

“Professor,” Harry said, “if the worst happens in a case like that, is there any way of \emph{maintaining} the Transfiguration?”

“No,” Professor McGonagall said flatly. “Sustaining a Transfiguration is a constant drain on your magic which scales with the size of the target form. And you would need to recontact the target every few hours, which is, in a case like this, impossible. Disasters like this are \emph{unrecoverable!}”

Professor McGonagall leaned forwards, her face very hard. “You will absolutely never under any circumstances Transfigure anything into a liquid or a gas. No water, no air. Nothing like water, nothing like air. Even if it is not meant to drink. Liquid \emph{evaporates,} little bits and pieces of it get into the air. You will not Transfigure anything that is to be burned. It will make smoke and someone could breathe that smoke! You will never Transfigure anything that could conceivably go inside anyone’s body by any means. No food. Nothing that \emph{looks like} food. Not even as a funny little prank where you mean to tell them about your mud pie before they actually eat it. You will never do it. Period. Inside this classroom or out of it or \emph{anywhere.} Is that well understood by \emph{every single student?}”

“Yes,” said Harry, Hermione, and a few others. The rest seemed to be speechless.

“\emph{Is that well understood by every single student?}”

“Yes,” they said or muttered or whispered.

“If you break any of these rules you will not further study Transfiguration during your stay at Hogwarts. Repeat along with me. I will never Transfigure anything into a liquid or gas.”

“I will never Transfigure anything into a liquid or gas,” said the students in ragged chorus.

“Again! Louder! I will never Transfigure anything into a liquid or gas.”

“I will never Transfigure anything into a liquid or gas.”

“I will never Transfigure anything that looks like food or anything else that goes inside a human body.”

“I will never Transfigure anything that is to be burned because it could make smoke.”

“You will never Transfigure anything that looks like money, including Muggle money,” said Professor McGonagall. “The goblins have ways of finding out who did it. As a matter of recognised law, the goblin nation is in a permanent state of \emph{war} with all magical counterfeiters. They will not send Aurors. They will send an army.”

“I will never Transfigure anything that looks like money,” repeated the students.

“And \emph{above all},” said Professor McGonagall, “you will not Transfigure any living subject, \emph{especially yourselves.} It will make you very sick and possibly dead, depending on how you Transfigure yourself and how long you maintain the change.” Professor McGonagall paused. “Mr~Potter is currently holding up his hand because he has seen an Animagus transformation—specifically, a human transforming into a cat and back again. But an Animagus transformation is not \emph{free} Transfiguration.”

Professor McGonagall took a small chunk of wood out of her pocket. With a tap of her wand it became a glass ball. Then she said “\emph{Crystferrium!}” and the glass ball became a steel ball. She tapped it with her wand one last time and the steel ball became a piece of wood once more. “\emph{Crystferrium} transforms a subject of solid glass into a similarly shaped target of solid steel. It cannot do the reverse, nor can it transform a desk into a pig. The most general form of Transfiguration—free Transfiguration, which you will be learning here—is capable of transforming any subject into any target, at least so far as physical form is concerned. For this reason, free Transfiguration must be done wordlessly. Using Charms would require different words for every different transformation between subject and target.”

Professor McGonagall gave her students a sharp look. “\emph{Some} teachers begin with Transfiguration Charms and move on to free Transfiguration afterwards. Yes, that would be much easier in the beginning. But it can set you in a poor mould which impairs your abilities later. Here you will learn free Transfiguration from the \emph{very start}, which requires that you cast the spell wordlessly, by holding the subject form, the target form, and the transformation within your own mind.”

“And to answer Mr~Potter’s question,” Professor McGonagall went on, “it is \emph{free} Transfiguration which you must never do to any living subject. There are Charms and potions which can safely, reversibly transform living subjects in \emph{limited} ways. An Animagus with a missing limb will still be missing that limb after transforming, for example. Free Transfiguration is \emph{not} safe. Your body will change while it is Transfigured—breathing, for example, results in a constant loss of the body’s stuff to the surrounding air. When the Transfiguration wears off and your body tries to revert to its \emph{original} form, it will not quite be able to do so. If you press your wand to your body and imagine yourself with golden hair, afterwards your hair will fall out. If you visualise yourself as someone with clearer skin, you will be taking a long stay at St. Mungo’s. And if you Transfigure yourself into an adult bodily form, then, when the Transfiguration wears off, you will die.”

That explained why he had seen such things as fat boys, or girls less than perfectly pretty. Or old people, for that matter. That wouldn’t happen if you could just Transfigure yourself every morning…Harry raised his hand and tried to signal Professor McGonagall with his eyes.

“\emph{Yes}, Mr~Potter?”

“Is it possible to Transfigure a living subject into a target that is static, such as a coin—no, excuse me, I’m terribly sorry, let’s just say a steel ball.”

Professor McGonagall shook her head. “Mr~Potter, even inanimate objects undergo small internal changes over time. There would be no visible changes to your body afterwards, and for the first minute, you would notice nothing wrong. But in an hour you would be sick, and in a day you would be dead.”

“Erm, excuse me, so if I’d read the first chapter I could have \emph{guessed} that the desk was originally a desk and not a pig,” Harry said, “but only if I made the \emph{further} assumption that you didn’t want to kill the pig, that might \emph{seem} highly probable but—”

“I can foresee that marking your tests will be an endless source of delight to me, Mr~Potter. But if you have other questions can I please ask you to wait until after class?”

“No further questions, professor.”

“Now repeat after me,” said Professor McGonagall. “I will never try to Transfigure any living subject, especially myself, unless specifically instructed to do so using a specialised Charm or potion.”

“If I am not sure whether a Transfiguration is safe, I will not try it until I have asked Professor McGonagall or Professor Flitwick or Professor Snape or the Headmaster, who are the only recognised authorities on Transfiguration at Hogwarts. Asking another student is \emph{not} acceptable, even if they say that they remember asking the same question.”

“Even if the current Defence Professor at Hogwarts tells me that a Transfiguration is safe, and even if I see the Defence Professor do it and nothing bad seems to happen, I will not try it myself.”

“I have the absolute right to refuse to perform any Transfiguration about which I feel the slightest bit nervous. Since not even the Headmaster of Hogwarts can order me to do otherwise, I certainly will not accept any such order from the Defence Professor, even if the Defence Professor threatens to deduct one hundred House points and have me expelled.”

“If I break any of these rules I will not further study Transfiguration during my time at Hogwarts.”

“We will repeat these rules at the start of every class for the first month,” said Professor McGonagall. “And now, we will begin with matches as subjects and needles as targets…put away your wands, thank you, by ‘begin’ I meant that you will begin taking notes.”

Half an hour before the end of class, Professor McGonagall handed out the matches.

At the end of the class Hermione had a silvery-looking match and the entire rest of the class, Muggleborn or otherwise, had exactly what they’d started with.

Professor McGonagall awarded her another point for Ravenclaw.

\later

After the Transfiguration class was dismissed, Hermione came over to Harry’s desk as Harry was putting away his books into his pouch.

“You know,” Hermione said with an innocent expression on her face, “I earned two points for Ravenclaw today.”

“So you did,” Harry said shortly.

“But that wasn’t as good as your \emph{seven} points,” she said. “I guess I’m just not as intelligent as you.”

Harry finished feeding his homework into the pouch and turned to Hermione with his eyes narrowed. He’d actually forgotten about that.

She \emph{batted her eyelashes} at him. “We have lessons every day, though. I wonder how long it will take you to find some more Hufflepuffs to rescue? Today is Monday. So that gives you until Thursday.”

The two of them stared into each other’s eyes, unblinking.

Harry spoke first. “Of course you realise this means war.”

“I didn’t know we’d been at peace.”

All of the other students were now watching with fascinated eyes. All of the other students, plus, unfortunately, Professor McGonagall.

“Oh, Mr~Potter,” sang Professor McGonagall from the other side of the room, “I have some good news for you. Madam Pomfrey has approved your suggestion for preventing breakage in her Spimster wickets, and the plan is to finish the job by the end of next week. I’d say that deserves…let’s call it ten points for Ravenclaw.”

Hermione’s face was gaping in betrayal and shock. Harry imagined his own face didn’t look much different.

“\emph{Professor…}” Harry hissed.

“Those ten points are \emph{unquestionably} deserved, Mr~Potter. I would not hand out House points on a whim. To you it might have been a simple matter of seeing something fragile and suggesting a way to protect it, but Spimster wickets are expensive, and the Headmaster was \emph{not} pleased the last time one broke.” Professor McGonagall looked thoughtful. “My, I wonder if any other student has ever earned seventeen House points on his first day of lessons. I’ll have to look it up, but I suspect that’s a new record. Perhaps we should have an announcement at dinner time?”

“\scream{Professor!}” Harry shrieked. “This is \emph{our} war! Stop meddling!”

“Now you have until Thursday of \emph{next} week, Mr~Potter. Unless, of course, you engage in some sort of mischief and \emph{lose} House points before then. Addressing a professor disrespectfully, for example.” Professor McGonagall put a finger on her cheek and looked reflective. “I expect you’ll hit negative numbers before the end of Friday.”

Harry’s mouth snapped shut. He sent his best Death Glare at McGonagall but she only seemed to find it amusing.

“Yes, definitely an announcement at dinner,” Professor McGonagall mused. “But it wouldn’t do to offend the Slytherins, so the announcement should be brief. Just the number of points and the fact of the record…and if anyone comes to you for help with their schoolwork and is disappointed that you haven’t even started reading your textbooks, you can always refer them to Miss~Granger.”

“\emph{Professor!}” said Hermione in a rather high-pitched voice.

Professor McGonagall ignored her. “My, I wonder how long it will take before Miss~Granger does something deserving of a dinner time announcement? I look forward to seeing it, whatever it may be.”

Harry and Hermione, by unspoken mutual consent, turned and stormed out of the classroom. They were followed by a trail of hypnotised Ravenclaws.

“Um,” Harry said. “Are we still on for after dinner?”

“Of course,” said Hermione. “I wouldn’t want you to fall further behind on your studying.”

“Why, thank you. And let me say that as brilliant as you are already, I can’t help but wonder what you’ll be like once you have some elementary training in rationality.”

“Is it really that useful? It didn’t seem to help you with Charms or Transfiguration.”

There was a slight pause.

“Well, I only got my schoolbooks four days ago. That’s why I had to earn those seventeen House points without using my wand.”

“Four days ago? Maybe you can’t read eight books in four days but you might have at least read \emph{one}. How many days will it take to finish at that rate? You know all that mathematics, so can you tell me what’s eight, times four, divided by zero?”

“I’ve got classes now, which you didn’t, but weekends are free, so… limit of eight times four divided by epsilon as epsilon approaches zero plus…10:47\AM on Sunday.”

“I did it in \emph{three} days actually.”

“2:47\PM on Saturday it is, then. I’m sure I’ll find the time somewhere.”

And there was evening and there was morning, the first day.
