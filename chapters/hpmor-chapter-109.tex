\chapter{Reflections}

\lettrinemph{E}{ven} \emph{the
greatest artifact can be defeated by a counter-artifact that is lesser, but
specialized.}

That was what the Defence Professor had told Harry, after dropping the True
Cloak of Invisibility to pool in fuliginous folds near Harry's shoes.

\emph{The Mirror of Perfect Reflection has power over what is reflected within
it, and that power is said to be unchallengeable. But since the True Cloak of
Invisibility produces a perfect absence of image, it should evade this
principle rather than challenging it.}

There had followed a series of questions in Parseltongue establishing that
Harry currently did not intend to do anything stupid or try to run away, and
further reminders that Professor Quirrell could sense him and had spells to
detect the Cloak and was holding hostage hundreds of lives plus Hermione.

Then Harry was told to don the Cloak, open the door that lay beyond the
quenched fires, and advance through the door into the final chamber; as
Professor Quirrell stood well back, outside of that door's sight.

The last chamber was illuminated in lights of soft gold, and the stone walls
were of gentle white and faced with marble.

In the centre of the room stood a simple and unornamented golden frame, and
within the frame was a portal to another gold-illuminated room, beyond whose
door which lay another Potions chamber; that was what Harry's brain told him.
The Mirror's transformation of light was so perfect that conscious thought was
required to deduce that the room inside the frame was only a reflection, rather
than a portal. (Though it might have been easier to intuit if Harry hadn't been
invisible, just then.)

The Mirror did not touch the ground; the golden frame had no feet. It didn't
look like it was hovering; it looked like it was fixed in place, more solid and
more motionless than the walls themselves, like it was nailed to the reference
frame of the Earth's motion.

"Is the Mirror there? Is it moving?" came Professor Quirrell's commanding voice
from the Potions Chamber.

"\parsel{Iss there,}" Harry hissed back. "\parsel{Not moving.}"

Again tones of command rang forth. "Walk around to the back of the Mirror."

From behind, the golden frame appeared solid, showing no reflections, and Harry
said so in Parseltongue.

"Now take off your Cloak," commanded Professor Quirrell's voice still from
within the Potions room. "Report to me at once if the Mirror moves to face you."

Harry took off his Cloak.

The Mirror remained nailed to the reference frame of Earth's motion; and Harry
reported this.

Shortly after there came a hissing and seething, and a balefire phoenix melted
through the marble wall behind Harry, the ambient light in the room taking on a
red tinge as it entered. Professor Quirrell followed behind it, walking out of
the new-made corridor that had been carved, his black formal shoes unharmed by
the red-glowing molten surface beneath. "Well," Professor Quirrell said, "that
is one possible trap averted. And now{\ldots}" Professor Quirrell exhaled. "Now
we will think of possible strategies for retrieving the Stone from the Mirror,
and you will try them; for I prefer not to let my own image be reflected. I
give you fair warning, this is the part that may prove tedious."

"I take it this isn't a problem you can solve with Fiendfyre?"

"Ha," said Professor Quirrell, and gestured.

The balefire phoenix moved forward in a rush of crimson terror, the red light
casting writhing shadows on the remaining marble walls. Harry jumped back
before he could think.

The dreadful dark-red blaze rushed past Professor Quirrell, surged into the
golden back of the Mirror, and disappeared as fast as it touched the gold.

Then the fire was gone, and the room was tinged scarlet no more.

There was no scratch upon the golden surface, no glow to mark the absorption of
heat. The Mirror had simply remained in place, untouched.

Chills went down Harry's spine. If he'd been playing Dungeons and Dragons and
the dungeon master had reported that result, Harry would have suspected a
mental illusion, and rolled to disbelieve.

Upon the centre of the golden back had appeared a sequence of runes in no known
alphabet, black absences of light in small lines and curves, arranged in a
level horizontal row. The thought occurred to Harry that some minor concealing
illusion had been consumed in the Fiendfyre, a far lesser enchantment that had
been added to prevent children from seeing those letters{\ldots}

"How old is this Mirror?" Harry said in almost a whisper.

"Nobody knows, Mr.~Potter." The Defence Professor reached out his fingers
toward the runes, a look of something like reverence on his face; but his
fingers did not touch the gold. "But my guess is the same as yours, I think. It
is said, in certain legends that may or may not be fabrications, that this
Mirror reflects \emph{itself} perfectly and therefore its existence is
absolutely stable. So stable that the Mirror was able to survive when every
other effect of Atlantis was undone, all its consequences severed from Time.
You can see why I was amused when you suggested Fiendfyre." The Defence
Professor let his hand fall.

Even in the middle of everything else, Harry felt the awe, if that was true.
The golden frame gleamed no brighter than before, for all the revelation; but
you could imagine it going back, and back, into a civilization that had been
made to never be{\ldots} "What---does the Mirror \emph{do,} exactly?"

"An excellent question," said Professor Quirrell. "The answer is in the runes
that are written upon the Mirror's golden back. Read them to me."

"They're not in any alphabet I recognize. They look like randomly oriented
chicken-scratches drawn by Tolkien elves."

"Read them anyway. \parsel{Iss not dangerouss.}"

"The runes say, \emph{noitilov detalo partxe tnere hoc ruoy tu becafruoy ton wo
hsi---}" Harry stopped, feeling more prickles at his spine.

Harry knew what the rune for noitilov \emph{meant}. It meant noitilov. And the
next runes said to detalo the noitilov until it reached partxe, then keep the
part that was both tnere and hoc. That belief felt like knowledge, like he could
have answered `Yes' with confident authority if somebody asked him
whether the ton wo was ruoy or becafruoy. It was just that when Harry tried to
relate those concepts to any other concepts, he drew a blank.

"\parsel{Do you undersstand what wordss mean, boy?}"

"\parsel{Don't think sso.}"

Professor Quirrell gave a soft exhalation, his eyes not leaving the golden
frame. "I had wondered if perhaps the Words of False Comprehension might be
understandable to a student of Muggle science. Apparently not."

"Maybe---" Harry began.

\emph{Really, Ravenclaw?} said Slytherin. \emph{You're pulling this NOW?}

"Maybe I could try again to understand the words if I knew more about the
Mirror?" said Harry's Ravenclaw part, which had assumed direct control.

Professor Quirrell's lips quirked up. "As with most ancient things, scholars
have written down enough lies that it is hard to be sure of anything by now. It
is definite that the Mirror is at least as old as Merlin, for it is known that
Merlin used it as a tool. It is also known that after his death, Merlin left
written instructions that the Mirror did not need to be sealed away, despite it
having certain powers that might normally cause one to worry. He wrote that,
given how painstakingly the Mirror had been crafted to not destroy the world,
it would be easier to destroy the world using a lump of cheese."

This statement struck Harry as not entirely reassuring.

"Certain other facts about the Mirror are attested by famous wizards who were
reasonably skeptical, and whose word has otherwise proven reliable. The
Mirror's most characteristic power is to create alternate realms of existence,
though these realms are only as large in size as what can be seen within the
Mirror; it is known that people and other objects can be stored therein. It is
claimed by several authorities that the Mirror alone of all magics possesses a
true moral orientation, though I am not sure what that could mean in practical
terms. I would expect moralists to call the Cruciatus Curse by their name of
`evil' and the Patronus Charm by their name of `good'; I cannot guess what a
moralist would think was any \emph{more} moral than that. But it is claimed,
for example, that phoenixes came into our world from a realm that was evoked
inside this Mirror."



Words like \emph{Jeepers} and what his parents would have termed inappropriate
language were all running through Harry's head, none very coherently, as he
stared at the golden back of the Mirror.

"I have wandered the world and encountered many stories that are not often
heard," said Professor Quirrell. "Most of them seemed to me to be lies, but a
few had the ring of history rather than storytelling. Upon a wall of metal in a
place where no one had come for centuries, I found written the claim that some
Atlanteans foresaw their world's end, and sought to forge a device of great
power to avert the inevitable catastrophe. If that device had been completed,
the story claimed, it would have become an absolutely stable existence that
could withstand the channeling of unlimited magic in order to grant wishes. And
also---this was said to be the vastly harder task---the device would somehow
avert the inevitable catastrophes any sane person would expect to follow from
that premise. The aspect I found interesting was that, according to the tale
writ upon those metal plates, the rest of Atlantis ignored this project and
went upon their ways. It was sometimes praised as a noble public endeavour, but
nearly all other Atlanteans found more important things to do on any given day
than help. Even the Atlantean nobles ignored the prospect of somebody other
than themselves obtaining unchallengeable power, which a less experienced cynic
might expect to catch their attention. With relatively little support, the tiny
handful of would-be makers of this device labored under working conditions that
were not so much dramatically arduous, as pointlessly annoying. Eventually time
ran out and Atlantis was destroyed with the device still far from complete. I
recognise certain echoes of my own experience that one does not usually see
invented in mere tales." A twist in the dry smile. "But perhaps that is merely
my own preference for one tale among a hundred other legends. You perceive,
however, the echo of Merlin's statement about the Mirror's creators shaping it
to not destroy the world. Most importantly for our purposes, it may explain why
the Mirror would have the previously unknown capability that Dumbledore or
Perenelle seems to have evoked, of showing any person who steps before it an
illusion of a world in which one of their desires has been fulfilled. It is the
sort of sensible precaution you can imagine someone building into a
wish-granting creation meant to not go horribly wrong."

"Wow," Harry whispered, and meant it. This was Magic with a capital M, the sort
of Magic that appeared in \emph{So You Want To Be A Wizard,} not just a
collection of random physics-violating things you could do with a wand.

Professor Quirrell gestured at the golden back. "The final property upon which
most tales agree, is that whatever the unknown means of commanding the
Mirror---of that Key there are no plausible accounts---the Mirror's
instructions cannot be shaped to react to individual people. So it is not
possible for Perenelle to command this Mirror, `only give the Stone to
Perenelle'. Dumbledore cannot state, `Only give the Stone to one who wishes to
give it to Nicholas Flamel'. There is in the Mirror a blindness such as
philosophers have attributed to ideal justice; it must treat all who come
before it by the same rule, whatever rule may be in force. Thus, there must be
some rule for reaching the Stone's hiding-place which anyone can invoke. And
now you see why \emph{you}, called the Boy-Who-Lived, shall implement whatever
strategies the two of us devise. For it was said that this thing possesses a
moral orientation, and it may have been given commands reflecting the same. I
am well aware that on conventional terms you are said to be Good, just as I am
said to be Evil." Professor Quirrell smiled, rather darkly. "So as our first
attempt---though not our last, rest assured---let us see what this Mirror makes
of your attempt to retrieve the Stone in order to save the life of Hermione
Granger and hundreds of your fellow students."

"And the \emph{first} version of that plan," said Harry, who was beginning to
finally understand, "the one you invented on Friday in my first week of
Hogwarts, called for the Stone to be retrieved by Dumbledore's golden child,
the Boy-Who-Lived, making a selfless and noble attempt to save the life of his
dying Defence teacher, Professor Quirrell."

"Of course," said Professor Quirrell.

It was a poetical sort of plot, Harry supposed, but his appreciation of that
elegance was being hampered by the surrounding circumstances.

Then another thought occurred to Harry.

"Um," Harry said. "You think that this Mirror is a trap for you---"

"There is no way beneath the heavens that it is not meant as a trap."

"That is to say, it's a trap for Lord Voldemort. Only it can't be a trap for
him personally. There has to be a general rule that underlies it, some
generalizable quality of Lord Voldemort that triggers it." Without conscious
awareness, Harry was frowning hard at the Mirror's golden back.

"As you say," said Professor Quirrell, who was beginning to frown at Harry's
frowning.

"Well, on the first Thursday of this year, the mad Headmaster Dumbledore, who
I'd just seen incinerate a chicken, told me that I had no chance whatsoever of
getting into his forbidden corridor, since I didn't know the spell
\emph{Alohomora.}"

"I \emph{see,}" said Professor Quirrell. "Oh, dear. I wish you had thought to
mention this to me a good deal earlier."

Neither of them needed to state aloud the obvious, that this bit of reverse
reverse psychology had successfully ensured that Harry would stay the heck away
from Dumbledore's forbidden corridor.

Harry was still concentrating. "Do you think Dumbledore suspects that I am, in
his terms, a horcrux of Lord Voldemort, or more generally, that some aspects of
my personality were copied off Lord Voldemort?" Even as Harry asked this aloud,
he realized what a dumb question it was, and how much completely blatant
evidence he'd already seen that---

"Dumbledore cannot \emph{possibly} have missed it," said Professor Quirrell.
"It is not exactly subtle. What else is Dumbledore to think, that you are an
actor in a play whose stupid author has never met a real eleven-year-old? Only
a gibbering dullard would believe that---ah, never mind."

The two of them stared at the Mirror in silence.

Finally Professor Quirrell sighed. "I have outwitted myself, I fear. Neither
you nor I dare be reflected in this Mirror. I suppose I must command Professor
Sprout to undo my Obliviations of Mr.~Nott and Miss Greengrass{\ldots} You see,
the other great difficulty of the Mirror is that the rule by which it treats
those reflected will disregard external forces, such as False Memories or a
Confundus Charm. The Mirror reflects only those forces arising from within the
person themselves, the states of mind they arrive at through their own choices;
so it is said in several places. That is why I had Mr.~Nott and Miss
Greengrass, believing different stories about why the Stone's extraction was
necessary, ready to appear before this Mirror." Professor Quirrell rubbed at
the bridge of his nose. "I constructed other stories for other students, ready
for me to set into motion with the chosen trigger{\ldots} but as this day
approached, I began to feel pessimistic about the project. Such as Nott and
Greengrass still seem worth trying, if we cannot think of something better. But
I wonder if Dumbledore has tried to construct this puzzle to specifically
resist Voldemort's cunning. I wonder if he might have succeeded. If you devise
an alternative plan which I approve enough to try, \parsel{I promisse that
whatever pawn I ssend forth sshall not be harmed by me, then or ever; nor do I
expect to break that promisse}. And I remind you again of the hostages I hold
to my failure, both Miss Granger and all the others."

Again they stared at the mirror in silence, the elder Tom Riddle and the
younger.

"I suspect, Professor," Harry said after a time, "that your entire class of
hypotheses about somebody needing to want the Stone for good or honest purposes
is mistaken. The Headmaster wouldn't set a retrieval rule like that."

"Why?"

"Because Dumbledore knows how easy it is to end up believing that you're doing
the right thing when you're actually not. It'd be the first possibility he
imagined."

"\parsel{Iss it truth or trickery that I hear?}"

"\parsel{Am being honesst,}" Harry said.

Professor Quirrell nodded. "Then your point is well taken."

"I'm not sure why you think this puzzle is solvable," Harry said. "Just set a
rule like, your left hand must hold a small blue pyramid and two large red
pyramids, and your right hand must be squeezing mayonnaise onto a hamster---"

"No," Professor Quirrell said. "No, I think not. The legends are unclear on
what rules can be given, but I think it must have something to do with the
Mirror's original intended use---it must have something to do with the deep
desires and wishes arising from within the person. Squeezing mayonnaise onto a
hamster will not qualify as that, for most people."

"Huh," Harry said. "Maybe the rule is that the person has to not want to use
the Stone at all---no, that's too easy, the story you gave Mr.~Nott solves it."

"In some ways you may understand Dumbledore better than I," said Professor
Quirrell. "So now I ask you this: how would Dumbledore use his notion of the
acceptance of death to guard this Stone? For that above all he thinks I cannot
comprehend, and he is not far wrong."

Harry thought about this for a while, considering several ideas and discarding
them. And then, having thought of something, Harry considered remaining
silent{\ldots} before mapping out the obvious part of the future conversation
where Professor Quirrell asked him to say in Parseltongue if he'd thought of
something.

Reluctantly, Harry spoke. "Would Dumbledore think that this Mirror could reach
the afterlife? Could he put the Stone into something that he \emph{thinks} is
an afterlife, so that only people who believe in an afterlife can see it?"

"Hm{\ldots}" Professor Quirrell said. "Possibly{\ldots} yes, there is a certain
plausibility to it. Using this setting of the Mirror to show people their
heart's desires{\ldots} Albus Dumbledore would see himself reunited with his
family. He would see himself united with them \emph{in death,} wanting to die
himself rather than wishing for them to be returned to life. His brother
Aberforth, his sister Ariana, his parents Kendra and Percival{\ldots} it would
be Aberforth to whom Dumbledore gave the Stone, I think. Would the Mirror
recognize that Aberforth particularly had been given the Stone? Or will any
person's dead relative do, if that person believes their relative's spirit
would give them back the Stone?" Professor Quirrell was pacing in a short
circle, keeping well away from Harry and the Mirror as he moved. "But all this
is only one idea. Let us devise another."

Harry began to tap his cheek, then stopped abruptly as he realized where he'd
picked up that gesture. "What if Perenelle is the one who put the Stone in
here? Maybe she keyed the Mirror to give the Stone only to the person who put
it in originally."

"Perenelle has lived this long by knowing her limitations," said Professor
Quirrell. "She does not overestimate her own intellect, she is not prideful, if
that were so she would have lost the Stone long ago. Perenelle will not try to
think of a good Mirror-rule herself, not when Master Flamel can leave the
matter in Dumbledore's wiser hands{\ldots} but the rule of only returning the
Stone to the one who remembers placing it, also works if Dumbledore himself has
placed the Stone. It would be a hard rule to bypass, since I cannot simply
Confund someone into believing that they put in the Stone{\ldots} I would have
to create a false Stone, and a false Mirror, and arrange the drama{\ldots}"
Professor Quirrell was frowning, now. "But it is still something that
Dumbledore would imagine Voldemort being able to arrange, given time. If at all
possible, Dumbledore will want to make the key to the Mirror a state of mind he
thinks I \emph{cannot} arrange in a pawn---or a rule that Dumbledore thinks
Voldemort can never comprehend, such as a rule involving the acceptance of
one's own death. That is why I considered your previous idea plausible."

Then Harry had an idea.

He was not sure if it was a good idea.

{\ldots}it wasn't like Harry had a lot of choice here.

"Arguendo," Harry said. "We're not sure what's necessary to retrieve the Stone.
But a \emph{sufficient} condition should involve Albus Dumbledore, or maybe
someone else, in a state of mind where they believe that the Dark Lord has been
defeated, that the threat is over, and that it is time to take out the Stone
and give it back to Nicholas Flamel. We aren't sure which part of that person's
state of mind, let's say Dumbledore's, will be the necessary part that he
thinks Lord Voldemort can't understand or duplicate; but under those conditions
Dumbledore's entire state of mind will be \emph{sufficient.}"

"Reasonable," said Professor Quirrell. "So?"

"The corresponding strategy," Harry said carefully, "is to mimic Dumbledore's
state of mind under those conditions, in as much detail as possible, while
standing in front of the mirror. And this state of mind must have been produced
by internal forces, not external ones."

"But how are we to get that without Legilimency or the Confundus Charm, both of
which would certainly be external---ha. I \emph{see.}" Professor Quirrell's
ice-pale eyes were suddenly piercing. "You suggest that I Confund
\emph{myself,} as you cast that hex upon yourself during your first day in
Battle Magic. So that it is an internal force and not an external one, a state
of mind that comes about through only my own choices. Say to me whether you
have made this suggestion with the intention of trapping me, boy. Say it to me
in Parseltongue."

"\parsel{My mind that you assked to devisse sstrategy may perhapss have been
influenced by ssuch an intent---who knowss? Knew you would be ssusspiciouss,
assk thiss very question. Decission is up to you, teacher. I know nothing you
do not know, about whether thiss iss likely to trap you. Do not call it
betrayal by me if you choosse thiss for yoursself, and it failss.}" Harry felt
a strong impulse to smile, and suppressed it.

"Lovely," said Professor Quirrell, who \emph{was} smiling. "I suppose there are
some threats from an inventive mind that even questioning in Parseltongue
cannot neutralize."
\later
Harry put on the Cloak of Invisibility, at Professor Quirrell's orders, to
\parsel{sstop the man who sshall believe himsself to be sschoolmasster from
sseeing you,} as Professor Quirrell said in Parseltongue.

"Wearing the Cloak or no, you will stand in range of the Mirror yourself,"
Professor Quirrell said. "If a gush of lava comes forth, you will also burn. I
feel that much symmetry should apply."

Professor Quirrell pointed to a spot near the right of the door through which
they'd entered the room, before the Mirror and well back of it. Harry, wearing
the Cloak, went to where Professor Quirrell had pointed him, and did not argue.
It was increasingly unclear to Harry whether both Riddles dying here would be a
bad thing, even with hundreds of other student hostages at stake. For all of
Harry's good intentions, he'd mostly shown himself so far to be an idiot, and
the returned Lord Voldemort was a threat to the entire world.

(Though either way, Harry couldn't see Dumbledore doing the lava thing.
Dumbledore was probably sufficiently angry at Voldemort to discard his usual
restraint, but lava wouldn't permanently stop an entity that Dumbledore
believed to be a discorporate soul.)

Then Professor Quirrell pointed with his wand, and a shimmering circle appeared
around where Harry was standing on the floor. This, Professor Quirrell said,
would soon become a Greater Circle of Concealment, by which nothing within that
circle could be heard or seen from the outside. Harry would not be able to make
himself apparent to the false Dumbledore by taking off the Cloak, nor by
shouting.

"You \emph{will not} cross this circle once it is active," Professor Quirrell
said. "That would cause you to touch my magic, and while Confunded I might not
remember how to halt the resonance that would destroy us both. And further,
since I do not want you throwing shoes---" Professor Quirrell made another
gesture, and just within the Greater Circle of Concealment, a slight shimmer
appeared in the air, a globe-shaped distortion. "\parsel{Thiss barrier will
explode if touched, by you or other material thing.} The resonance might lash
at me afterward, but you would also be dead. Now tell me in Parseltongue that
you do not intend to cross this circle or take off your Cloak or do
\emph{anything} at all impulsive or stupid. Tell you me you will wait quietly
here, under the Cloak, until this is over."

This Harry repeated back.

Then Professor Quirrell's robes became black tinged with gold, such robes as
Dumbledore might wear upon a formal occasion; and Professor Quirrell pointed
his own wand at his head.

Professor Quirrell stayed motionless for a long time, still holding his wand to
his head. His eyes were closed in concentration.

And then Professor Quirrell said, "\emph{Confundus.}"

At once the expression of the man standing there changed; he blinked a few
times as though confused, lowering his wand.

A deep weariness spread over the face Professor Quirrell had worn; without any
visible change his eyes seemed older, the few lines in his face calling
attention to themselves.

His lips were set in a sad smile.

Without any hurry, the man quietly walked over to the Mirror, as though he had
all the time in the world.

He crossed into the Mirror's range of reflection without anything happening,
and stared into the surface.

What the man might be seeing there, Harry could not tell; to Harry it seemed
that the flat, perfect surface still reflected the room behind it, like a
portal to another place.

"Ariana," breathed the man. "Mother, father. And you, my brother, it is done."

The man stood still, as if listening.

"Yes, done," the man said. "Voldemort came before this mirror, and was trapped
by Merlin's method. He is only one more sealed horror now."

Again the listening stillness.

"I would that I could obey you, my brother, but it is better this way." The man
bowed his head. "He is denied his death, forever; that vengeance is terrible
enough."

Harry felt a twinge, watching this, a sense that this was \emph{not} what
Dumbledore would have said, it seemed more like a strawman, a shallow
stereotype{\ldots} but then this wasn't the real Aberforth's spirit either,
this was who Professor Quirrell imagined Dumbledore imagined Aberforth was, and
that doubly-reflected image of Aberforth wouldn't notice anything amiss{\ldots}

"It is time to give back the Philosopher's Stone," said the man who thought he
was Dumbledore. "It must go back into Master Flamel's keeping, now."

Listening stillness.

"No," said the man, "Master Flamel has kept it safe these many years from all
who would seek immortality, and I think it will be safest in his hands{\ldots}
no, Aberforth, I do think his intentions are good."

Harry couldn't control the tension that was running through him like a live
wire; he was having trouble breathing. Imperfect, Professor Quirrell's
Confundus Charm had been imperfect. The underlying personality of Professor
Quirrell was leaking through and seeing the obvious question: why was it okay
for Nicholas Flamel himself to have the Stone if immortality was so awful? Even
if Professor Quirrell imagined Dumbledore was blind to the question,
he hadn't included a clause in the Confundus saying that
\emph{Dumbledore's image of Aberforth} wouldn't think of it; and all of this
was ultimately a reflection of Professor Quirrell's own mind, an image from
within the intelligence of Tom Riddle{\ldots}

"Destroy it?" said the man. "Maybe. I am not sure it \emph{can} be destroyed,
or Master Flamel would have done it long since. I think, many times, that he
has regretted making it{\ldots} Aberforth, I promised him, and we are not so
ancient or so wise ourselves. The Philosopher's Stone must go back into the
keeping of the one who made it."

And Harry's breath stopped.

The man was holding an irregular chunk of scarlet glass in his left hand, the
size perhaps of Harry's thumb from fingernail to the first joint. The sheened
surface of the scarlet glass made it seem wet; the appearance was of blood,
suspended in time and made into a jagged surface.

"Thank you, my brother," the man said quietly.

\emph{Is that what the Stone should look like? Does Professor Quirrell know
what the true Stone should look like? Will the Mirror give back the real Stone
under these conditions, or make an imitation and return that?}

And then---

"No, Ariana," the man said, smiling gently, "I fear I must go now. Be patient,
my dearest, it will be soon enough that I join you in truth{\ldots} why? Why, I
am not sure why I must go{\ldots} when I hold the Stone I am to step aside from
the Mirror and wait for Master Flamel to contact me, but I am not sure why I
need to step aside from the Mirror to do that{\ldots}" The man sighed. "Ah, I
am getting old. It is well this dreadful war ended when it did. I suppose there
is no harm if I speak to you for a time, my dearest, if you wish it so."

A headache was starting behind Harry's eyes; some part of Harry was trying to
send a message about not having breathed in a while, but no one was listening.
\emph{Imperfect}, Professor Quirrell's Confundus Charm had been imperfect,
Professor Quirrell's image of Dumbledore's image of Ariana wanted to talk to
Dumbledore, and maybe didn't want to wait because Professor Quirrell knew on
some level that there wasn't really an afterlife, and the previously implanted
impulse to leave after getting the Stone \emph{wasn't standing up to
Riddle-Ariana's arguments{\ldots}}

And then Harry felt himself become very calm. He started breathing again.

Either way, there wasn't much Harry could do about it. Professor Quirrell had
stopped Harry from intervening; well, Professor Quirrell was welcome to reap
the consequences of that decision. If the consequences caught Harry as well, so
be it.

The man who thought he was Dumbledore was mostly nodding patiently, sometimes
replying to his dearest sister. Sometimes the man cast an uneasy look to one
side; as if feeling a strong impulse to go, but suppressing that impulse with
the great patience and politeness and concern for his sister that Professor
Quirrell imagined Albus Dumbledore having.

Harry saw it the instant the Confundus wore off, and the man's expression
changed, becoming again the face of Professor Quirrell.

And in the same instant the Mirror changed, no longer showing Harry the
reflection of the room, showing instead the form of the real Albus Dumbledore,
as though he were standing just behind the Mirror and visible through it.

The real Dumbledore's face was set, and grim.

"Hello, Tom," said Albus Dumbledore.
